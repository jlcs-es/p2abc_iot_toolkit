%*****************************************
\chapter{State of the art}\label{ch:stateoftheart}
%*****************************************

In this chapter we present the two domains of this project: the Zero-Knowledge Proof based systems and the Internet of Things.


\section{Anonymous Credentials Systems}
% ZKP vs minimal disclosure como OAuth
% Persiano
% U -Prove
% Idemix
% P2ABCE 
% TODO: explicar P2ABCE
%\ac{P2ABCE}
% Credential: a certified container of attributes where an attribute has a type and a value

%%%
Other solutions are based on minimal disclosure. Standards like OAuth offer secure delegated access to the user information and when registering to a new service, the user can give a key to access only the data they want from another trusted service. This lets the service provider to work with the OAuth server, offering the same service as before without knowing as much data.

But this is only minimizes how many services have our data. Our OAuth provider could be attacked, revealing all our data, or our service provider, revealing now less data.
%%%

%%%%

A Quick Introduction to
Anonymous Credentials
Gregory Neven
IBM Zürich Research Laboratory
August 2008


Intuitively, an anonymous credential can be thought of as a digital signature by the Issuer on a list of
attribute-value pairs, e.g. the list
(fname=”Alice”, lname=”Anderson”, bdate=”1977/05/10”, nation=”DE”).
The most straightforward way for the User to convince a Verifier of her list of attributes would be to
simply transmit her credential to the Verifier. This approach has a number of disadvantages, most
notably
· that the User has to reveal all of her attributes so that the Verifier can check the signature;
· that the Verifier can reuse the credential to impersonate Alice wrt other Verifiers.
With anonymous credentials, the User never transmits the credential itself, but rather uses it to convince
the Verifier that her attributes satisfy certain properties – without leaking anything about the credential
other than the shown properties. This has the obvious advantage that the Verifier can no longer reuse the
credential to impersonate Alice. Another advantage is that anonymous credentials allow the User to
reveal a selected subset of her attributes. In the example above, Alice could for example reveal her first
name and last name, but keep her birth date hidden.
Stronger even, apart from showing the exact value of an attribute, the User can even convince the
Verifier that some complex predicate over the attributes holds. In particular, she can demonstrate any
predicate consisting of the following operations:
· Integer arithmetic: att+c , att+att , att·c : sum of attributes and/or constants, product of
attributes with constants
· Comparative: exp1 = exp2 , exp1 < exp2 , exp1 > exp2 , exp1  exp2 , exp1  exp2 : compare
arithmetic expressions over attributes/constants
· Logical: pred1 AND pred2 , pred1 OR pred2
Note that revealing a subset of attributes comes down to the special of showing that the predicate
att1 = val1 AND att2 = val2 AND …
holds. In the example above, Alice could show that she’s an overage national of an EU country by
showing that he has a credential satisfying
bdate  today – 18y AND ( nation=”DE” OR nation=”FR” OR … ) .
The Show protocol does not leak any information other than the truth of the statement. So in the
example above, not only does Alice’s name remain hidden from the Verifier, but so do her exact birth
date and nationality; the only information leaked is that her birth date is more than 18 years ago, and that
her nationality is one of the EU countries.
%%%

%%%%%%%%%%%%%%%%%%%%%%%%

@book{book:185217,
	title =     {Financial Cryptography: 8th International Conference, FC 2004, Key West, FL, USA, February 9-12, 2004. Revised Papers},
	author =    {Jack R. Selby (auth.), Ari Juels (eds.)},
	publisher = {Springer-Verlag Berlin Heidelberg},
	isbn =      {3540224203,9783540224204,9783540278092},
	year =      {2004},
	series =    {Lecture Notes in Computer Science 3110},
	edition =   {1},
	volume =    {},
	pages = {196-211},
	url =       {http://gen.lib.rus.ec/book/index.php?md5=24CD58AE88D5564A03C2E44B96732298}}


Persiano and Visconti presented a non-transferable anonymous
credential system that is multi-show and for which it is possible to prove
properties (encoded by a linear Boolean formula) of the credentials. Unfortunately,
their proof system is not efficient since the step in which a user proves
possession of credentials (that needs a number of modular exponentiations that
is linear in the number of credentials) must be repeated times (where is the
security parameter) in order to obtain a satisfying soundness

%%%%%%%%%%%%%%%%%%%%%%%%%


%%%%%%%%%
@book{book:947508,
	title =     {Security and Privacy in Communication Networks: 7th International ICST Conference, SecureComm 2011, London, UK, September 7-9, 2011, Revised Selected Papers},
	author =    {Zhiyun Qian, Z. Morley Mao, Ammar Rayes, David Jaffe (auth.), Muttukrishnan Rajarajan, Fred Piper, Haining Wang, George Kesidis (eds.)},
	publisher = {Springer-Verlag Berlin Heidelberg},
	isbn =      {978-3-642-31908-2,978-3-642-31909-9},
	year =      {2012},
	series =    {Lecture Notes of the Institute for Computer Sciences, Social Informatics and Telecommunications Engineering 96},
	edition =   {1},
	volume =    {},
	pages = {243-260},
	url =       {http://gen.lib.rus.ec/book/index.php?md5=A7667DB3EE5B36AAA6E93AE07E888D8D}}

U-Prove. Stefan Brands provided the first integral description of the U-Prove
technology in his thesis [6] in 2000, after which he founded the company Credentica
in 2002 to implement and sell this technology. Microsoft acquired Credentica
in 2008 and published the U-Prove protocol specification [5] in 2010
under the Open Specification Promise4 together with open source reference software
development kits (SDKs) in C# and Java.
The U-Prove technology is centred around a so-called U-Prove token. This
token serves as a pseudonym for the prover. It contains a number of attributes
which can be selectively disclosed to a verifier. Hence the prover decides which
attributes to show and which to withhold (e.g. one can reveal the birth date,
but not the residence address). Besides the attributes the token contains two
information fields, one defined by the issuer, and one by the prover. These fields
are always disclosed and can be used to provide some meta data such as a validity
date of the token. Finally there is the token’s public-key, which aggregates all
information in the token, and a signature from the issuer over this public-key to
ensure the authenticity

5. Brands, S., Paquin, C.: U-Prove cryptographic specification v1.0. Tech. rep., Microsoft
Corporation (March 2010)
6. Brands, S.A.: Rethinking Public Key Infrastructures and Digital Certificates:
Building in Privacy. MIT Press (August 2000)
%%%%%%%%%%%





%% Idemix:

En \citep{Camenisch:GroupSig} \citep{Camenisch:AnonCred} \citep{camenisch2002signature} describen los métodos criptográficos iniciales utilizando ZKPs que evolucionarán a Idemix. En \citep{idemixSpec} se encuentra la especificación actual del protocolo Identity Mixer de IBM.

\section{Internet of Things}

% Adaptar seguridad "clásica" de cifrado de mensajes con claves simétrica y asimétrica, RSA, curva elíptica, DH, 
% Que el principio de las ZKP ya se usaba en las tarjetas, de manera básica, pero no como parte de un sistema mayor como Idemix
% Vanets y otros intentos de meter sistemas previos en IoT

Regarding the Internet of Things security, 


\section{Idemix}

\paragraph{TODO: Move to motivation}

%%%%
The problem of Internet security has been approached by securing the transmission channel (e.g. SSL/TLS) and the data stored in both ends (strict access policies, local encryption, etc.). In the end, the data exists in two entities, the owner of the data and the service provider. The owner is the most interested in securing his data, and can apply as many measures as he wants, but only on his side of the table. The service provider that stores the user data needs it to provide the service, and a successful attack would reveal many users data, aside from how many measures each one used to protect it. The case of PlayStation Network outage in 2011 \citep{PSN2011} affected 77 million accounts, with suspected credit card fraud, is an example of this kind of attacks.

\hfil

%%%%


%%%%%%%%%%%%%%

Estudio teórico de cómo integrar Idemix en una smart card, según el reparto de cálculos y valores secretos \citep{luuk}.
Implementación de Idemix en smart cards MULTOS \citep{vullers2013efficient}
\citep{sucasas} Diseño de sistema anonymous credential system with sc support, y haciendo pruebas en simulador de smart card, sin embargo no utilizan Idemix ni dan toda su funcionalidad, pero sí aportan un estudio de cómo afrontar la implementación: smart cards.
Integración en IoT device con P2ABCE en \citep{vanet}





Why ABC4Trust Card Lite: soporte de las APDUs del proyecto P2ABCE, más grande, multi-engine, como dice su powerpoint, adaptable a cualquiera que use Discrete Log.