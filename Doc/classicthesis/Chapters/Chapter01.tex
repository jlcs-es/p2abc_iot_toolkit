%************************************************
\chapter{Introduction}\label{ch:introduction}
%************************************************

In recent years some new concepts have appeared in common people's vocabulary, like \textit{machine learning}, \textit{big data}, \textit{artificial intelligence}, \textit{automation}, etc., but there are two in particular that we are going to focus and try to combine: \ac{IoT} and Internet Security \& Privacy.

The \ac{IoT} is a term with a wide range of interpretations \citep{Atzori20102787}, but a breaf definition could be the set of devices, mainly resource constrained, that are interconnected between them in order to achieve a goal. This includes from lampposts with proximity sensors that talk to each other in order to light up part of the street when a passerby walks by, to a sensor on your clothes that tells the washing machine how much detergent to use.

Security \& Privacy, thanks to organizations like \href{https://wikileaks.org/}{WikiLeaks}, are now taken in consideration by any technology consumer, not only professionals. People are conscious about what their data can be used for, demanding more control over it.

And IoT has proved to not address neither security nor privacy, with recent events like the Mirai botnet DDoS attack on October 2016, considered the biggest DDoS in history \citep{jeyanthi:2017}, or like the multiple vulnerabilities affecting baby monitors \citep{rapid7babycam}.


A recent approach to address the problem of privacy is the \textit{strong anonymity},  that conceals our personal details while letting us continue to operate online as a clearly defined individual \citep{stronganonymity}. One very promising way to achieve this is using \acp{ZKP}
\marginpar{To read more about \acs{ZKP} aside the introduction done in this thesis, you can read my Mathematics thesis \citep{TODO}.}
, cryptographic methods that allows to proof knowledge of data without disclosing it. Furthermore, IBM has been developing a cryptographic protocol suite for privacy-preserving authentication and transfer of certified attributes based on \ac{ZKP}, called Identity Mixer, Idemix for short \citep{idemixurl}.

The goal of this project is to integrate Idemix with the \ac{IoT}. It will be done using the ABC4Trust's \ac{P2ABCE}, a framework that defines common architecture, policy language and data artifacts, but based on either IBM's Idemix or Microsoft's U-Prove \citep{p2abcurl}. This gives us a standardized language to exchange Idemix's messages between \ac{IoT} devices and usual PCs.

\section{Motivation}

\section{Challenges}

\section{Goals}

\section{Outline of this thesis}




